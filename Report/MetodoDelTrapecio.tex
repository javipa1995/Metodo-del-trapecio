%%%%%%%%%%%%%%%%%%%%%%%%%%%%%%%%%%%%%%%%%%%%%%%%%%%%%%%%%%%%%%%%%%%%%%%%%%%%%%%%%%%%%%%%%%%%%%%%%%%%%%
% Plantilla básica de Latex en Español.
%
% Autor: Andrés Herrera Poyatos (https://github.com/andreshp) 
%
% Es una plantilla básica para redactar documentos. Utiliza el paquete fancyhdr para darle un
% estilo moderno pero serio.
%
% La plantilla se encuentra adaptada al español.
%
%%%%%%%%%%%%%%%%%%%%%%%%%%%%%%%%%%%%%%%%%%%%%%%%%%%%%%%%%%%%%%%%%%%%%%%%%%%%%%%%%%%%%%%%%%%%%%%%%%%%%%

%-----------------------------------------------------------------------------------------------------
%	INCLUSIÓN DE PAQUETES BÁSICOS
%-----------------------------------------------------------------------------------------------------

\documentclass{article}


%-----------------------------------------------------------------------------------------------------
%	SELECCIÓN DEL LENGUAJE
%-----------------------------------------------------------------------------------------------------

% Paquetes para adaptar Látex al Español:
\usepackage[spanish,es-noquoting, es-tabla, es-lcroman]{babel} % Cambia 
\usepackage[utf8]{inputenc}                                    % Permite los acentos.
\selectlanguage{spanish}                                       % Selecciono como lenguaje el Español.

%-----------------------------------------------------------------------------------------------------
%	SELECCIÓN DE LA FUENTE
%-----------------------------------------------------------------------------------------------------

% Fuente utilizada.
\usepackage{courier}                    % Fuente Courier.
\usepackage{microtype}                  % Mejora la letra final de cara al lector.

%-----------------------------------------------------------------------------------------------------
%	ESTILO DE PÁGINA
%-----------------------------------------------------------------------------------------------------

% Paquetes para el diseño de página:
\usepackage{fancyhdr}               % Utilizado para hacer títulos propios.
\usepackage{lastpage}               % Referencia a la última página. Utilizado para el pie de página.
\usepackage{extramarks}             % Marcas extras. Utilizado en pie de página y cabecera.
\usepackage[parfill]{parskip}       % Crea una nueva línea entre párrafos.
\usepackage{geometry}               % Asigna la "geometría" de las páginas.

% Se elige el estilo fancy y márgenes de 3 centímetros.
\pagestyle{fancy}
\geometry{left=3cm,right=3cm,top=3cm,bottom=3cm,headheight=1cm,headsep=0.5cm} % Márgenes y cabecera.
% Se limpia la cabecera y el pie de página para poder rehacerlos luego.
\fancyhf{}

% Espacios en el documento:
\linespread{1.1}                        % Espacio entre líneas.
\setlength\parindent{0pt}               % Selecciona la indentación para cada inicio de párrafo.

% Cabecera del documento. Se ajusta la línea de la cabecera.
\renewcommand\headrule{
	\begin{minipage}{1\textwidth}
	    \hrule width \hsize 
	\end{minipage}
}

% Texto de la cabecera:
\lhead{\subject}                          % Parte izquierda.
\chead{}                                    % Centro.
\rhead{\doctitle \ - \docsubtitle}              % Parte derecha.

% Pie de página del documento. Se ajusta la línea del pie de página.
\renewcommand\footrule{                                 
\begin{minipage}{1\textwidth}
    \hrule width \hsize   
\end{minipage}\par
}

\lfoot{}                                                 % Parte izquierda.
\cfoot{}                                                 % Centro.
\rfoot{Página\ \thepage\ de\ \protect\pageref{LastPage}} % Parte derecha.


%----------------------------------------------------------------------------------------
%   MATEMÁTICAS
%----------------------------------------------------------------------------------------

% Paquetes para matemáticas:                     
\usepackage{amsmath, amsthm, amssymb, amsfonts, amscd} % Teoremas, fuentes y símbolos.
     
 % Nuevo estilo para definiciones
 \newtheoremstyle{definition-style} % Nombre del estilo
 {5pt}                % Espacio por encima
 {0pt}                % Espacio por debajo
 {}                   % Fuente del cuerpo
 {}                   % Identación: vacío= sin identación, \parindent = identación del parráfo
 {\bf}                % Fuente para la cabecera
 {.}                  % Puntuación tras la cabecera
 {.5em}               % Espacio tras la cabecera: { } = espacio usal entre palabras, \newline = nueva línea
 {}                   % Especificación de la cabecera (si se deja vaía implica 'normal')
 
 % Nuevo estilo para teoremas
 \newtheoremstyle{theorem-style} % Nombre del estilo
 {5pt}                % Espacio por encima
 {0pt}                % Espacio por debajo
 {\itshape}           % Fuente del cuerpo
 {}                   % Identación: vacío= sin identación, \parindent = identación del parráfo
 {\bf}                % Fuente para la cabecera
 {.}                  % Puntuación tras la cabecera
 {.5em}               % Espacio tras la cabecera: { } = espacio usal entre palabras, \newline = nueva línea
 {}                   % Especificación de la cabecera (si se deja vaía implica 'normal')
 
 % Nuevo estilo para ejemplos y ejercicios
 \newtheoremstyle{example-style} % Nombre del estilo
 {5pt}                % Espacio por encima
 {0pt}                % Espacio por debajo
 {}                   % Fuente del cuerpo
 {}                   % Identación: vacío= sin identación, \parindent = identación del parráfo
 {\scshape}                % Fuente para la cabecera
 {:}                  % Puntuación tras la cabecera
 {.5em}               % Espacio tras la cabecera: { } = espacio usal entre palabras, \newline = nueva línea
 {}                   % Especificación de la cabecera (si se deja vaía implica 'normal')
 
 % Teoremas:
 \theoremstyle{theorem-style}  % Otras posibilidades: plain (por defecto), definition, remark
 \newtheorem{theorem}{Teorema}[section]  % [section] indica que el contador se reinicia cada sección
 \newtheorem{corollary}[theorem]{Corolario} % [theorem] indica que comparte el contador con theorem
 \newtheorem{lemma}[theorem]{Lema}
 \newtheorem{proposition}[theorem]{Proposición}
 
 % Definiciones, notas, conjeturas
 \theoremstyle{definition}
 \newtheorem{definition}{Definición}[section]
 \newtheorem{conjecture}{Conjetura}[section]
 \newtheorem*{note}{Nota} % * indica que no tiene contador
 
 % Ejemplos, ejercicios
 \theoremstyle{example-style}
 \newtheorem{example}{Ejemplo}[section]
 \newtheorem{exercise}{Ejercicio}[section]

%-----------------------------------------------------------------------------------------------------
%	BIBLIOGRAFÍA
%-----------------------------------------------------------------------------------------------------

\usepackage[backend=bibtex, style=numeric]{biblatex}
\usepackage{csquotes}

\addbibresource{references.bib}

%-----------------------------------------------------------------------------------------------------
%	PORTADA
%-----------------------------------------------------------------------------------------------------

% Elija uno de los siguientes formatos.
% No olvide incluir los archivos .sty asociados en el directorio del documento.
\usepackage{title1}
%\usepackage{title2}
%\usepackage{title3}

%-----------------------------------------------------------------------------------------------------
%	TÍTULO, AUTOR Y OTROS DATOS DEL DOCUMENTO
%-----------------------------------------------------------------------------------------------------

% Título del documento.
\newcommand{\doctitle}{Ecuaciones diferenciales ordinarias}
% Subtítulo.
\newcommand{\docsubtitle}{Método del trapecio}
% Fecha.
\newcommand{\docdate}{1 \ de \ Enero \ de \ 2015}
% Asignatura.
\newcommand{\subject}{Métodos Numéricos II}
% Autor.
\newcommand{\docauthor}{Andrés Herrera Poyatos \\ Javier Poyatos Amador \\ Rodrigo Raya Castellano}
\newcommand{\docaddress}{Universidad de Granada}
\newcommand{\docemail}{}

%-----------------------------------------------------------------------------------------------------
%	RESUMEN
%-------------------------------					----------------------------------------------------------------------

% Resumen del documento. Va en la portada.
% Puedes también dejarlo vacío, en cuyo caso no aparece en la portada.
%\newcommand{\docabstract}{}
\newcommand{\docabstract}{En este texto puedes incluir un resumen del documento. Este informa al lector sobre el contenido del texto, indicando el objetivo del mismo y qué se puede aprender de él.}

\begin{document}

\maketitle

%-----------------------------------------------------------------------------------------------------
%	ÍNDICE
%-----------------------------------------------------------------------------------------------------

% Profundidad del Índice:
%\setcounter{tocdepth}{1}

\newpage
\tableofcontents
\newpage

%-----------------------------------------------------------------------------------------------------
%	SECCIÓN 1: MOTIVACIÓN
%-----------------------------------------------------------------------------------------------------

\section{Motivación: resolución de ecuaciones diferenciales ordinarias}

%-----------------------------------------------------------------------------------------------------
%	SECCIÓN 2: MÉTODO DEL TRAPECIO
%-----------------------------------------------------------------------------------------------------

\section{Descripción del método del trapecio}

	Considérese el problema de valores iniciales dado por la ecuación diferencial $y'(t) = f(t,y(t))$ sobre $[a,b]$ y la condición $y(a) = y_0$. Supóngase que se ha averiguado que el problema admite solución única. Se pretende obtener un método numérico para aproximar la imagen de $y$ en los puntos $t1, t2, \ldots, t_m \in [a,b]$. Si $t_0 = a$, utilizando la Proposición X, la solución de este PVI es la única solución de la siguiente ecuación
	
	\begin{equation}
		y(t)  = y(t_0) + \int_{t_0}^{t} f(s,y(s))) \ ds
	\end{equation}
	
	En este contexto se pueden aplicar los métodos de integración numérica para aproximar la integral que aparece en la segunda igualdad. Para ello supóngase que $f$ es diferenciable. En tal caso una obvia inducción concluye que $y$ es de clase infinito. Por tanto, se puede utilizar la fórmula del trapecio para integración numérica, obteniendo la siguiente igualdad
	
	\begin{equation}
		y(t_{n+1}) = y(t_n) + \frac{h}{2} \left[f(t_n,y(t_n)) + f(t_{n+1}, y(t_{n+1}))\right] - \frac{h^3}{12}y^{3)}(\xi_n)
	\end{equation}


	donde $h = t_{n+1}-t_n$. Esto proporciona la siguiente aproximación, que tiene error $- \frac{h^3}{12}y^{3)}(\xi_n)$,

	\begin{equation} \label{eq:app}
		y(t_{n+1}) \approx y(t_n) + \frac{h}{2} \left[f(t_n,y(t_n)) + f(t_{n+1}, y(t_{n+1}))\right]
	\end{equation}

	El problema reside en que para aproximar el valor de $y$ en $t_{n+1}$ se debe conocer previamente dicho valor. En este contexto se plantean dos soluciones diferentes obteniendo dos métodos, denominados método del trapecio explícito e implícito respectivamente. Para facilitar la redacción de estos métodos se denotará $w_n$ a la aproximación obtenida para $y(t_n)$.
	
	\subsection{Método del trapecio explícito}
		
		Recuérdese en este punto el método de Euler para ecuaciones diferenciales ordinarias. Este método consiste en aproximar $y(t_{n+1})$ a partir de $y(t_n)$ de la siguiente forma
		
		\begin{equation}
			y(t_{n+1}) \approx w_{n+1} = w_n + h f(t_n,w_n))
		\end{equation}

		Esto es, se predice $y(t_{n+1})$ como el valor que toma en $t_{n+1}$ la recta tangente a $y$ en $t_n$. Este es un método sencillo con error.....

		Se puede combinar el método de Euler con (\ref{eq:app}) para obtener la siguiente aproximación

		\begin{equation} \label{eq:app-exp}
			y(t_{n+1}) \approx w_{n+1} = w_n + \frac{h}{2} \left[f(t_n,w_n) + f(t_{n+1}, w_n + h f(t_n,w_n))\right]
		\end{equation}

		Si 
		
		obteniendo el denominado método del trapecio explícito. 
  
  Esto es, si se conoce un valor de alguna solución al problema, la expresión previa proporciona un mecanismo para aproximar 

	En particular, se puede tomar $t_{n+1} > t_n$ y obtener la siguiente expresión:
	
	$$ y(t_{n+1})  = y(t_n) + \int_{t_n}^{t_{n+1}} f(s,y(s))) \ ds $$
	
	si se conoce  se puede aproximar un número finito de valores de la función 
que se pretende resolver para el problema de valores iniciales $y(a) = y_0$
Supóngase que el problema tiene solución única.

	\begin{figure}[h]
		\centering
		\includegraphics[width=10cm]{./Images/trapecio-vs-euler.png}
		\caption{Esquema visual del método del trapecio.} 
		\label{fig:trapecio-vs-euler}
	\end{figure}

%-----------------------------------------------------------------------------------------------------
%	SECCIÓN X: ERROR
%-----------------------------------------------------------------------------------------------------


\section{Estudio del error. Convergencia}


%-----------------------------------------------------------------------------------------------------
%	SECCIÓN X: ESTABILIDAD
%-----------------------------------------------------------------------------------------------------

\section{Estabilidad}

%-----------------------------------------------------------------------------------------------------
%	SECCIÓN X: CONCLUSIÓN
%-----------------------------------------------------------------------------------------------------

\section{Conclusión}


%-----------------------------------------------------------------------------------------------------
%	SECCIÓN X: REFERENCIAS
%-----------------------------------------------------------------------------------------------------

\printbibliography


\end{document}