\section{Método del trapecio}

	\subsection{Introducción}

		\begin{frame}{Introducción al método del trapecio}
			\fontsize{11}{11}\selectfont				
			\begin{proposition} \label{prop:sol-eq}
				Sea un PVI con $y'(t) = f(t,y(t))$ y $y(t_0) = y_0$.  Son equivalentes:
				\begin{enumerate}
					\item $y$ es una solución del PVI.
					\item $y(t) = y_0 + \int_{t_0}^{t} f(s,y(s))) ds \ \forall t \in [a,b]$
				\end{enumerate}
			\end{proposition}

			\kern 7mm
			\begin{tcolorbox}[colback=ChetwodeBlue!10,colframe=ChetwodeBlue!60]
				Nuestra solución verifica:
				\begin{equation*}
					y(t_1)  = y_0 + \int_{t_0}^{t_1} f(s,y(s))) \ ds
				\end{equation*}		
			\end{tcolorbox}
		\end{frame}

		\begin{frame}{Introducción al método del trapecio}
			\fontsize{10}{10}\selectfont				
			\begin{tcolorbox}[colback=ChetwodeBlue!10,colframe=ChetwodeBlue!60]
				Nuestra solución verifica:
				\begin{equation*}
				y(t_1)  = y_0 + \int_{t_0}^{t_1} f(s,y(s))) \ ds
				\end{equation*}		
			\end{tcolorbox}

			\begin{tcolorbox}[colback=ChetwodeBlue!10,colframe=ChetwodeBlue!60]
				\centering
				\textbf{Idea: Método del trapecio para integración numérica}
				\begin{itemize}
					\begin{equation} \label{eq:trapecio-igualdad}
						y(t_{1}) = y_0 + \frac{h}{2} \left[f(t_0,y_0) + f(t_1, y(t_1))\right] - \frac{h^3}{12}y^{3)}(\xi)
					\end{equation}
	
					\begin{equation} \label{eq:app}
						y(t_1) \approx w_1 = w_0 + \frac{h}{2} \left[f(t_0,w_0) + f(t_1, y(t_1))\right]
					\end{equation}			
				\end{itemize}

			\end{tcolorbox}
			
		\end{frame}


		\begin{frame}{¿Cómo funciona map reduce?}
			
			\fontsize{7}{8}\selectfont	
			\centering
			
			\noindent\makebox[\linewidth]{\rule{\textwidth}{0.4pt}}			
			\textbf{Algoritmo:} Obtención del número de ocurrencias de cada una de las palabras de un texto.
			\noindent\makebox[\linewidth]{\rule{\textwidth}{0.4pt}}

			\kern2mm					
			\includegraphics[width=0.88\textwidth]{./Images/count-words.png}		


			\noindent\makebox[\linewidth]{\rule{\textwidth}{0.4pt}}

			\kern2mm					
			\includegraphics[width=0.9\textwidth]{./Images/MapReduce.jpg}
		\end{frame}


		\begin{frame}{¿Cómo funciona map reduce?}
			\fontsize{6}{8}\selectfont
			\centering
			\begin{tikzpicture}
			\node (img1) {\includegraphics[width=0.9\textwidth]{./Images/MapReduce-MasterWorkers.jpg}};
			\node [above left = 2.5cm and 2cm of $(img1)$] (me) {\normalsize\textbf{Maestro - Esclavo}};
			\node [below = 3mm of $(me)$] (maestro) { \textbf{Maestro} $\twopartdef{{\color{TurkishRose}{\textbullet}}\  \text{Organiza los esclavos}}{{\color{TurkishRose}{\textbullet}}\ \text{Tolerancia a fallos}}$ };
			\node [below = 3mm of $(maestro)$] (esclavo) { \textbf{Esclavos} $\twopartdef{{\color{TurkishRose}{\textbullet}}\ Mapper \ (M)}{{\color{TurkishRose}{\textbullet}}\ Reducer  \ (R)}$ };
			\node [fill=TurkishRose!80,inner sep=2pt, text width=1.6cm, align=center, above right= 3cm and 3cm of $(img1)$] (pe) {Particionamiento de la entrada};
			\node [fill=ChetwodeBlue!80, inner sep=4.8pt, text width=1.4cm, align=center, below = 2.7mm of $(pe)$] (map) {Mappers};
			\node [fill=Jaguar!85, inner sep=2pt, text width=1.6cm, align=center, below = 2mm of $(map)$] (of) {\color{white}{Ordenación y filtrado}};
			\node [fill=ChetwodeBlue!80, inner sep=4.8pt, text width=1.4cm, align=center, below = 2.8mm of $(of)$] (re) {Reducers};
			\node [fill=TurkishRose!80,inner sep=2pt, text width=1.6cm, align=center, below = 2mm of $(re)$] (gs) {Generación de la salida};
			\path[draw=ChetwodeBlue!80,line width=2pt,->] ([xshift=3mm, yshift=3mm]pe.east) -- ([xshift=3mm, yshift=-3mm]gs.east);
			\end{tikzpicture}
		\end{frame}

	\subsection*{Software libre}
	
		\begin{frame}{}
			\kern-3mm
			\begin{center}
				\includegraphics[width=0.7\textwidth]{./Images/hadoop.png}
				\kern-2mm
				\begin{tcolorbox}[colback=ChetwodeBlue!10,colframe=ChetwodeBlue!60]
					\fontsize{8}{8}\selectfont
					\centering
					\textbf{Open-source software for reliable, scalable, distributed computing}
				\end{tcolorbox}
			\end{center}
	
			\centering
			\kern-3mm
			\fontsize{6}{8}\selectfont
			\begin{itemize}
				\item HDFS: Sistema de archivos distribuido basado en Google File System (GFS).
				\item YARN: Gestión de tareas, recursos y nodos.
				\item SPARK: Map Reduce + procesamiento iterativo y en memoria.
			\end{itemize}

			\begin{tikzpicture}					
				\node (img1) {\includegraphics[width=0.8\textwidth]{./Images/hadoop-enviroment.png}};
				\node [above right = 0.5cm and 2cm of $(img1)$] (j) {\includegraphics[width=0.2\textwidth]{./Images/java.jpg}};

			\end{tikzpicture}
						
		\end{frame}
