%%%%%%%%%%%%%%%%%%%%%%%%%%%%%%%%%%%%%%%%%%%%%%%%%%%%%%%%%%%%%%%%%%%%
% Authors: A. Herrera-Poyatos, F. Herrera
% Tittle: Algoritmo memético equilibrado con diversificación voraz
% 							 CAEPIA 2015
%%%%%%%%%%%%%%%%%%%%%%%%%%%%%%%%%%%%%%%%%%%%%%%%%%%%%%%%%%%%%%%%%%%%

\section{Motivación}

{
	% Set the headline 
	\setbeamertemplate{headline}{
		\begin{beamercolorbox}[sep=4pt]{title} 
			\usebeamerfont{frametitle}{\color{ChetwodeBlue}Motivación: Un nuevo método.}
		\end{beamercolorbox}
	}
	
	\begin{frame}{Motivación}
		\begin{columns}
			\column{0.6\textwidth}
			\begin{itemize}
				\item Ecuaciones diferenciales ordinarias.
				\item ¿Existe solución y es única?
				\item Métodos de discretización.
				\item Método de Euler.		
			\end{itemize}
			
			\column{0.5\textwidth}
			
			\begin{figure}[h]
				\centering
				\includegraphics[width=4cm]{./Images/interpret-pvi.png}
					\caption{Representación del campo vectorial asociado a la ecuación logística $y'(t) = c y(t) (1 - y(t))$.} 
			\end{figure}
		\end{columns}			
	\end{frame}
}
		
{
	% Set the headline 
	\setbeamertemplate{headline}{
		\begin{beamercolorbox}[sep=10pt]{title} 
			\usebeamerfont{frametitle}{\color{ChetwodeBlue}Motivación: Método de Euler}
		\end{beamercolorbox}
	}
		
	\begin{frame}
			\begin{tcolorbox}[colback=ChetwodeBlue!10,colframe=ChetwodeBlue!60]
				\begin{equation}
					\begin{cases}
					w_0=y_0 \\
					h_{i} = t_{i+1} - t_i \\
					w_{i+1} = w_i + h_{i} f(t_i,w_i)
					\end{cases}
				\end{equation}
				\centering
				\fontsize{10}{8}\selectfont	
				\textbf{Método de Euler.}
				\fontsize{9}{8}\selectfont								
				\begin{itemize}
					
				\end{itemize}
			\end{tcolorbox}	
	\end{frame}
}

{
	% Set the headline 
	\setbeamertemplate{headline}{
		\begin{beamercolorbox}[sep=10pt]{title} 
			\usebeamerfont{frametitle}{\color{ChetwodeBlue}Motivación: Necesidad de tratar y analizar los datos}
		\end{beamercolorbox}
	}
	
	\begin{frame}{}
				
	\end{frame}
}